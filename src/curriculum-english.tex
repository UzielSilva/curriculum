% resume.tex
% vim:set ft=tex spell:

\documentclass[10pt,letterpaper]{article}
\usepackage[letterpaper,margin=0.25in]{geometry}
\usepackage[utf8]{inputenc}
\usepackage{mdwlist}
\usepackage[T1]{fontenc}
\usepackage{textcomp}
\usepackage{tgpagella}
\pagestyle{empty}
\setlength{\tabcolsep}{0em}

\title{Curriculum Vitae}

% indentsection style, used for sections that aren't already in lists
% that need indentation to the level of all text in the document
\newenvironment{indentsection}[1]%
{\begin{list}{}%
	{\setlength{\leftmargin}{#1}}%
	\item[]%
}
{\end{list}}

% opposite of above; bump a section back toward the left margin
\newenvironment{unindentsection}[1]%
{\begin{list}{}%
	{\setlength{\leftmargin}{-0.5#1}}%
	\item[]%
}
{\end{list}}

% format two pieces of text, one left aligned and one right aligned
\newcommand{\headerrow}[2]
{\begin{tabular*}{\linewidth}{l@{\extracolsep{\fill}}r}
	#1 &
	#2 \\
\end{tabular*}}

% make "C++" look pretty when used in text by touching up the plus signs
\newcommand{\CPP}
{C\nolinebreak[4]\hspace{-.05em}\raisebox{.22ex}{\footnotesize\bf ++}}

% and the actual content starts here
\begin{document}

\begin{center}
{\LARGE \textbf{Uziel Silva Espino}}

Pino Montezuma \# 29\ \ \textbullet
\ \ Los Pinos de Michoacan\ \ \textbullet
\ \ Morelia, Michoacan. Mexico 58057
\\
443-146-0935\ \ \textbullet
\ \ uziel.silva.espino@gmail.com
\end{center}

\hrule
\vspace{-0.9em}
\subsection*{Personal accounts}

\begin{indentsection}{\parindent}
\hyphenpenalty=1000
\begin{description*}
	\item[Skype:]
	uziel.silva.espino
	\item[GitHub:]
	UzielSilva
	\item[Koding:]
	uzielsilva
	
\end{description*}
\end{indentsection}

\hrule
\vspace{-0.4em}
\subsection*{General skills.}

\begin{indentsection}{\parindent}
\hyphenpenalty=1000
\begin{description*}
	\item[Languages:] 
    English (50\% -70\%).
    
    \item[Agile Methodologies:]
    Scrum.
	
\end{description*}
\end{indentsection}

\hrule
\vspace{-0.4em}
\subsection*{Development skills.}

\begin{indentsection}{\parindent}
\hyphenpenalty=1000
\begin{description*}
	\item[General skills:]
	Test Driven Development.
	
	\item[Unit Testing Tools:]
	NUnit,
	JUnit,
	Jasmine,
    Karma,
    Phantom.js.
	
	\item[Programming languages:]
	VB6,
	VB.Net,
	Java,
	JavaScript,
	C\#,
	C,
	F\#,
	HTML,
	PHP,
    Bash,
	Ruby. 
	
	\item[Continuous Integration:]
	CruiseControl.NET,
    Travis CI. 
	
	\item[Databases:]
	MySQL,
    SQLite.
	
	\item[Javascript frameworks:]
	Angular.js,
    D3.js,
    Restangular,
    JQuery.
    
    \item[Development tools:]
	Node.js,
    Grunt,
    Gulp,
    Bower,
    Yeoman,
    MSBuild.
    
    \item[Mobile hybrid frameworks:]
    PhoneGap/Cordova,
    Ionic Framework.
	
	\item[Project contributions:]
	Karel the Robot IDE in Java,
    Rounded Defence.
	
\end{description*}
\end{indentsection}

\hrule
\vspace{-0.4em}
\subsection*{Testing/Automation skills.}

\begin{indentsection}{\parindent}
\hyphenpenalty=1000
\begin{description*}

    \item[General skills:]
    Test Case Execution,
    Functional Testing,
    Exploratory Testing,
    Bug Fix Verification,
    Test Case Automation.
	
	\item[Testing tools:]
	TestStack.White,
	Selenium,
    Charles Proxy,
    Fiddler,
    Appium.
	
	\item[Code analysis:]
	StyleCop,
	Simian. 
\end{description*}
\end{indentsection}

\hrule
\vspace{-0.4em}
\subsection*{Experience.}

\begin{itemize}
	\parskip=0.1em


	\item
	\headerrow
		{\textbf{Personal Projects.}}
		{\emph{2009 -- Present}}
	\begin{itemize*}
		\item "Karel the Robot IDE". Compiler based in Java for an educational language 
        called Karel The Robot. You can find the repository here: https://github.com/UzielSilva/KOMI
		
		\item "Rounded Defence". Video game developed in C\# with Unity game engine 
        with the purpose of participate in a video game contest. It counts with a dynamic 
        system of levels in XML and it has an elaborated and attractive GUI. You can find the repository here:
		https://github.com/UzielSilva/RoundedDefence
	\end{itemize*}

	\item
	\headerrow
		{\textbf{Scio}}
		{\textbf{http://sciodev.com/}}
	\\
	\headerrow
		{\emph{Application Developer 1C}}
		{\emph{2013 -- 2015}}
	\begin{itemize*}
		\item Test Driven Development. Powerful practice for professional developers.
        It forces you to develop functional and readable, and it prevents common bugs, 
        also it helps with project documentation.
		\item Agile Methodologies, like Scrum. Scrum helps project planning to be more realistic.
		\item Use of repositories like Git and SVN. This is very important in develop teams for 
        maintain a common version of a project.
	    \item Use of Continuous Integration tools for automate several revision tasks.
	\end{itemize*}
    
    \item
	\headerrow
		{\textbf{Freelance}}
		{\emph{2015 -- Present}}
	\begin{itemize*}
		\item Node.js Environment. There are several Node.js applications that allow project 
        development to be more easily, with functions like live-reload, task running, 
        dependency resolver, and many others.
        \item Mobile Apps. I learned how hybrid apps work, and I develop several 
        projects related with that.
        \item Back-end development. I developed web services based on PHP language, 
        and with MySQL and SQLite databases.
	\end{itemize*}
    
    \item
	\headerrow
		{\textbf{uTest}}
		{\textbf{https://www.utest.com/}}
	\\
	\headerrow
		{\emph{Freelance Functional Tester(Silver Tester)}}
		{\emph{January 2016 -- Present}}
	\begin{itemize*}
		\item Test Cycle Methodology. In this page I did effective issue reports and test case executions.
   	\end{itemize*}

\end{itemize}


\hrule
\vspace{-0.4em}
\subsection*{Relevant Education}

\begin{itemize}
	\parskip=0.1em

	\item 
	\headerrow
		{\textbf{UMSNH (Universidad Michoacana de San Nicolas de Hidalgo).}}
		{\textbf{Morelia, Michoacan}}
	\\
	\headerrow
		{\emph{Facultad de Ciencias Fisico-Matematicas Luis Manuel Gutierrez Rivera}}
		{\emph{2012 -- Present}}

\end{itemize}

\end{document}

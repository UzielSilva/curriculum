% resume.tex
% vim:set ft=tex spell:

\documentclass[10pt,letterpaper]{article}
\usepackage[letterpaper,margin=0.75in]{geometry}
\usepackage[utf8]{inputenc}
\usepackage{mdwlist}
\usepackage[T1]{fontenc}
\usepackage{textcomp}
\usepackage{tgpagella}
\pagestyle{empty}
\setlength{\tabcolsep}{0em}

% indentsection style, used for sections that aren't already in lists
% that need indentation to the level of all text in the document
\newenvironment{indentsection}[1]%
{\begin{list}{}%
	{\setlength{\leftmargin}{#1}}%
	\item[]%
}
{\end{list}}

% opposite of above; bump a section back toward the left margin
\newenvironment{unindentsection}[1]%
{\begin{list}{}%
	{\setlength{\leftmargin}{-0.5#1}}%
	\item[]%
}
{\end{list}}

% format two pieces of text, one left aligned and one right aligned
\newcommand{\headerrow}[2]
{\begin{tabular*}{\linewidth}{l@{\extracolsep{\fill}}r}
	#1 &
	#2 \\
\end{tabular*}}

% make "C++" look pretty when used in text by touching up the plus signs
\newcommand{\CPP}
{C\nolinebreak[4]\hspace{-.05em}\raisebox{.22ex}{\footnotesize\bf ++}}

% and the actual content starts here
\begin{document}

\begin{center}
{\LARGE \textbf{Uziel Silva Espino}}

Pino Michoacán \# 60\ \ \textbullet
\ \ Los Pinos de Michoacán\ \ \textbullet
\ \ Morelia, Michoacán. México 58057
\\
452-117-7849\ \ \textbullet
\ \ uziel.silva.espino@gmail.com
\end{center}

\hrule
\vspace{-0.4em}
\subsection*{Personal accounts}

\begin{indentsection}{\parindent}
\hyphenpenalty=1000
\begin{description*}
	\item[Skype username:]
	uziel.silva.espino
	\item[GitHub:]
	UzielSilva
	\item[Koding:]
	uzielsilva
	
\end{description*}
\end{indentsection}

\hrule
\vspace{-0.4em}
\subsection*{Programming skills.}

\begin{indentsection}{\parindent}
\hyphenpenalty=1000
\begin{description*}
	\item[General skills:]
	Test Driven Development, English (40\% - 60\%). \\
	
	\item[Testing tools:]
	Canopy,
	Selenium,
	NUnit. \\
	
	\item[Languages:]
	Java,
	C\#,
	C,
	F\#,
	HTML,
	Ruby. \\
	
	\item[Continuous Integration:]
	CruiseControl.NET. \\
	
	\item[Open Source Contributions:]
	Karel the Robot IDE in Java,
	Canopy Framework. \\
	
\end{description*}
\end{indentsection}

\hrule
\vspace{-0.4em}
\subsection*{Experience.}

\begin{itemize}
	\parskip=0.1em


	\item
	\headerrow
		{\textbf{As Freelancer.}}
		{\textbf{}}
	\\
	\headerrow
		{\emph{}}
		{\emph{2009 -- 2013}}
	\begin{itemize*}
		\item I was made several little applications with some people. The most
		important was Karel the Robot IDE, an IDE for help some students in
		learning process of programming languages.
	\end{itemize*}

	\item
	\headerrow
		{\textbf{Scio}}
		{\textbf{http://sciodev.com/}}
	\\
	\headerrow
		{\emph{Software Developer}}
		{\emph{2013 -- Present}}
	\begin{itemize*}
		\item In Scio I learned about Test Driven Development, a powerful practice
		for professional programming. This practice make clean code and reduce time
		of bug fixing.
		Unit tests help too with documentation, and is more easy to understand
		applications with unit tests.
		\item Agile Metodologies, like Scrum. This technique allows to all members
		know the advances for all members of team. Scrum is very useful. If any member
		isn't working, in Scrum all members know this, and help if they have any difficult.
		\item Use of repositories like TFS, GitHub and SVN. And remote repositories.
	    \item Use of test tools and developing automated tests with F\# and Canopy. This
		practice helps in app development. So it can find some UI errors. Automated tests
	    helps testing common automated tasks. It allows to the developer work in other tasks
	    while test of UI is finished.
	    \item Use of Continuous Integration for automated Build, test and deploy.
	    With CruiseControl.NET we can check if commits of members of the team don't break
	    the solution. We can add several tasks in the CCNet configuration and for each
	    commit, CCNet will run these tasks.
	\end{itemize*}

\end{itemize}


\hrule
\vspace{-0.4em}
\subsection*{Relevant Education}

\begin{itemize}
	\parskip=0.1em

	\item 
	\headerrow
		{\textbf{UMSNH (Universidad Michoacana de San Nicolas de Hidalgo).}}
		{\textbf{Morelia, Michoacán}}
	\\
	\headerrow
		{\emph{Facultad de Ciencias Físico-Matemáticas Luis Manuel Gutierrez Rivera}}
		{\emph{2012 -- Present}}
	\begin{itemize*}
		\item In this university I'm currently coursing my bachelors degree as matematician.
	\end{itemize*}

\end{itemize}

\end{document}

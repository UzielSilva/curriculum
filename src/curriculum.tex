% resume.tex
% vim:set ft=tex spell:

\documentclass[10pt,letterpaper]{article}
\usepackage[letterpaper,margin=0.60in]{geometry}
\usepackage[utf8]{inputenc}
\usepackage{mdwlist}
\usepackage[T1]{fontenc}
\usepackage{textcomp}
\usepackage{tgpagella}
\pagestyle{empty}
\setlength{\tabcolsep}{0em}

% indentsection style, used for sections that aren't already in lists
% that need indentation to the level of all text in the document
\newenvironment{indentsection}[1]%
{\begin{list}{}%
	{\setlength{\leftmargin}{#1}}%
	\item[]%
}
{\end{list}}

% opposite of above; bump a section back toward the left margin
\newenvironment{unindentsection}[1]%
{\begin{list}{}%
	{\setlength{\leftmargin}{-0.5#1}}%
	\item[]%
}
{\end{list}}

% format two pieces of text, one left aligned and one right aligned
\newcommand{\headerrow}[2]
{\begin{tabular*}{\linewidth}{l@{\extracolsep{\fill}}r}
	#1 &
	#2 \\
\end{tabular*}}

% make "C++" look pretty when used in text by touching up the plus signs
\newcommand{\CPP}
{C\nolinebreak[4]\hspace{-.05em}\raisebox{.22ex}{\footnotesize\bf ++}}

% and the actual content starts here
\begin{document}

\begin{center}
{\LARGE \textbf{Uziel Silva Espino}}

Pino Michoacán \# 60\ \ \textbullet
\ \ Los Pinos de Michoacán\ \ \textbullet
\ \ Morelia, Michoacán. México 58057
\\
452-117-7849\ \ \textbullet
\ \ uziel.silva.espino@gmail.com
\end{center}

\hrule
\vspace{-0.4em}
\subsection*{Cuentas personales}

\begin{indentsection}{\parindent}
\hyphenpenalty=1000
\begin{description*}
	\item[Skype:]
	uziel.silva.espino
	\item[GitHub:]
	UzielSilva
	\item[Koding:]
	uzielsilva
	
\end{description*}
\end{indentsection}

\hrule
\vspace{-0.4em}
\subsection*{Habilidades de programación.}

\begin{indentsection}{\parindent}
\hyphenpenalty=1000
\begin{description*}
	\item[Habilidades generales:]
	Test Driven Development, Inglés (40\% - 60\%). 
	
	\item[Herramientas de pruebas:]
	Canopy,
	Selenium,
	NUnit. 
	
	\item[Lenguajes de programación:]
	Java,
	C\#,
	C,
	F\#,
	HTML,
	Ruby. 
	
	\item[Integración continua:]
	CruiseControl.NET. 
	
	\item[Contribuciones en proyectos:]
	Karel the Robot IDE en Java,
	Canopy Framework. 
	
\end{description*}
\end{indentsection}

\hrule
\vspace{-0.4em}
\subsection*{Experiencia.}

\begin{itemize}
	\parskip=0.1em


	\item
	\headerrow
		{\textbf{Proyectos personales.}}
		{\textbf{}}
	\\
	\headerrow
		{\emph{}}
		{\emph{2009 -- 2013}}
	\begin{itemize*}
		\item Desarrollé varios proyectos pequeños individualmente y con ayuda de
		otro estudiante. De nuestros proyectos el más significativo ha sido "Karel the 
		Robot IDE". En éste proyecto usamos por primera vez un repositorio remoto
		localizado en la dirección: https://github.com/UzielSilva/KOMI
	\end{itemize*}

	\item
	\headerrow
		{\textbf{Scio}}
		{\textbf{http://sciodev.com/}}
	\\
	\headerrow
		{\emph{Application Developer 1A}}
		{\emph{2013 -- Present}}
	\begin{itemize*}
		\item En Scio aprendí acerca del Test Driven Development, una poderosa práctica
		para programar profesionalmente. Ésta práctica nos obligab a desarrollar código
		funcional y mantenible, ésto a su vez reduce la probabilidad de encontrar bugs
		en nuestro código, y, por lo tanto, a dedicar tanto tiempo a arreglar los mismos.
		Las pruebas unitarias a su vez nos ayudan con la documentación, y tanto el cliente
		como el desarrollador pueden comprender qué parte del código es la que se encarga
		de cumplir una función en particular, y si ésta es la función deseada.
		\item Metodologías Ágiles, como Scrum. Ésta técnica permite a todos los miembros
		saber los avances de todo el equipo. Scrum es muy útil, ya que si a alguien se le
		dificulta alguna tarea, todo el equipo puede darse cuenta, y pueden ayudar para 
		que pueda salir adelante.
		Por otro lado, Scrum ayuda a la planificación realista de proyectos. Si en cada
		Scrum las tareas que se propusieron se han cumplido, eso significa que la 
		planificación es realista, pero si se están acumulando tareas que no se alcanzan 
		a cumplir, entonces es tiempo para poder hacer una nueva planeación y comunicar
		acerca de ésta situación al cliente.
		\item Uso de repositorios como Git y SVN. El control de código es muy importante 
		cuando se desarrolla una aplicación, ya que si algo crítico pasara con nuestro 
		proyecto, podemos regresar a una versión estable. Por otra parte, también nos ayudan
		las ramas y las fusiones cuando trabajamos en equipo, ya que de ésta manera no 
		estorbamos en el trabajo de alguien más.
	    \item Uso de herramientas de pruebas automatizacas como Canopy. Éstas herramientas
	    son muy útiles ya que para probar cambios en la UI no tenemos que invertir tiempo,
	    sólo debemos de correr nuestras pruebas automatizadas y éstas nos dirán si no
	    rompimos algo.
	    Éstas pruebas también son una garantía al cliente de que su producto es como él
	    esperaba.
	    \item Uso de herramientas de Integración Continua para automatizar las tareas comunes
	    para evaluar el código. Con CruiseControl.NET podemos evaluar si los commits de
	    los miembros del equipo no cumplen con las espectativas como para poder integrarlos 
	    a la solución. También por medio de CruiseControl podemos hacer tareas como
	    deployment si es que el commit cumple con lo esperado.
	\end{itemize*}

\end{itemize}


\hrule
\vspace{-0.4em}
\subsection*{Educación Relevante}

\begin{itemize}
	\parskip=0.1em

	\item 
	\headerrow
		{\textbf{UMSNH (Universidad Michoacana de San Nicolas de Hidalgo).}}
		{\textbf{Morelia, Michoacán}}
	\\
	\headerrow
		{\emph{Facultad de Ciencias Físico-Matemáticas Luis Manuel Gutierrez Rivera}}
		{\emph{2012 -- Present}}
	\begin{itemize*}
		\item En ésta universidad actualmente estoy cursando la carrera de Licenciatura en
		Ciencias Físico-Matemáticas.
	\end{itemize*}

\end{itemize}

\end{document}

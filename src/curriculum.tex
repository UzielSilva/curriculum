% resume.tex
% vim:set ft=tex spell:

\documentclass[10pt,letterpaper]{article}
\usepackage[letterpaper,margin=0.25in]{geometry}
\usepackage[utf8]{inputenc}
\usepackage{mdwlist}
\usepackage[T1]{fontenc}
\usepackage{textcomp}
\usepackage{tgpagella}
\pagestyle{empty}
\setlength{\tabcolsep}{0em}

% indentsection style, used for sections that aren't already in lists
% that need indentation to the level of all text in the document
\newenvironment{indentsection}[1]%
{\begin{list}{}%
	{\setlength{\leftmargin}{#1}}%
	\item[]%
}
{\end{list}}

% opposite of above; bump a section back toward the left margin
\newenvironment{unindentsection}[1]%
{\begin{list}{}%
	{\setlength{\leftmargin}{-0.5#1}}%
	\item[]%
}
{\end{list}}

% format two pieces of text, one left aligned and one right aligned
\newcommand{\headerrow}[2]
{\begin{tabular*}{\linewidth}{l@{\extracolsep{\fill}}r}
	#1 &
	#2 \\
\end{tabular*}}

% make "C++" look pretty when used in text by touching up the plus signs
\newcommand{\CPP}
{C\nolinebreak[4]\hspace{-.05em}\raisebox{.22ex}{\footnotesize\bf ++}}

% and the actual content starts here
\begin{document}

\begin{center}
{\LARGE \textbf{Uziel Silva Espino}}

Pino Michoacán \# 60\ \ \textbullet
\ \ Los Pinos de Michoacán\ \ \textbullet
\ \ Morelia, Michoacán. México 58057
\\
443-146-0935\ \ \textbullet
\ \ uziel.silva.espino@gmail.com
\end{center}

\hrule
\vspace{-0.4em}
\subsection*{Cuentas personales}

\begin{indentsection}{\parindent}
\hyphenpenalty=1000
\begin{description*}
	\item[Skype:]
	uziel.silva.espino
	\item[GitHub:]
	UzielSilva
	\item[Koding:]
	uzielsilva
	
\end{description*}
\end{indentsection}

\hrule
\vspace{-0.4em}
\subsection*{Habilidades de programación.}

\begin{indentsection}{\parindent}
\hyphenpenalty=1000
\begin{description*}
	\item[Habilidades generales:]
	Test Driven Development, Inglés (20\% - 40\%). 
	
	\item[Herramientas de pruebas:]
	Canopy,
	Fluent Automation,
	TestStack.White,
	Selenium,
	NUnit,
	JUnit,
	Jasmine.
	
	\item[Lenguajes de programación:]
	VB6
	VB.Net
	Java,
	JavaScript,
	C\#,
	C,
	F\#,
	HTML,
	Ruby. 
	
	\item[Integración continua:]
	CruiseControl.NET. 
	
	\item[Análisis de código:]
	StyleCop
	Simian. 
	
	\item[Bases de datos:]
	MySQL.
	
	\item[Javascript:]
	Angular.js
	
	\item[Contribuciones en proyectos:]
	Karel the Robot IDE en Java.
	
\end{description*}
\end{indentsection}

\hrule
\vspace{-0.4em}
\subsection*{Experiencia.}

\begin{itemize}
	\parskip=0.1em


	\item
	\headerrow
		{\textbf{Proyectos personales.}}
		{\textbf{}}
	\\
	\headerrow
		{\emph{}}
		{\emph{2009 -- Presente}}
	\begin{itemize*}
		\item "Karel the Robot IDE". En éste proyecto desarrollamos nuestro propio 
		compilador para el lenguaje Karel Pascal y Karel Java en Java, se puede encontrar el
		repositorio en la dirección: https://github.com/UzielSilva/KOMI
		
		\item "Rounded Defence". Es un videojuego desarrollado en C\# con el motor de Unity 
		con el motivo de ser presentado para un concurso de programación de videojuegos.
		Cuenta con un sistema dinámico de niveles en XML y tiene una interfaz gráfica bastante
		trabajada, se puede encontrar el repositorio en la dirección: 
		https://github.com/UzielSilva/RoundedDefence
	\end{itemize*}

	\item
	\headerrow
		{\textbf{Scio}}
		{\textbf{http://sciodev.com/}}
	\\
	\headerrow
		{\emph{Application Developer 1C}}
		{\emph{2013 -- 2015}}
	\begin{itemize*}
		\item Test Driven Development. Es una poderosa práctica	para programar profesionalmente
		que nos obliga a desarrollar código	funcional y mantenible, y reducir la probabilidad 
		de encontrar errores. Además nos ayuda con la documentación del código.
		\item Metodologías Ágiles, como Scrum. Ésta técnica permite a todos los miembros
		saber los avances de todo el equipo.  Scrum ayuda a la planificación realista de 
		proyectos. 
		\item Uso de repositorios como Git y SVN. El control de código es muy importante 
		cuando se desarrolla una aplicación, al brindar la posibilidad de regresar a una
		versión anterior del código. Por otra parte, las ramas y las fusiones ayudan cuando 
		se trabaja en equipo.
	    \item Uso de herramientas de Integración Continua para automatizar las tareas comunes
	    para evaluar el código. Con CruiseControl.NET podemos evaluar si el código de
	    los miembros del equipo cumplen con las espectativas como para poder integrarlos 
	    a la solución. También por medio de CruiseControl podemos hacer tareas automatizadas.
	\end{itemize*}

\end{itemize}


\hrule
\vspace{-0.4em}
\subsection*{Educación Relevante}

\begin{itemize}
	\parskip=0.1em

	\item 
	\headerrow
		{\textbf{UMSNH (Universidad Michoacana de San Nicolas de Hidalgo).}}
		{\textbf{Morelia, Michoacán}}
	\\
	\headerrow
		{\emph{Facultad de Ciencias Físico-Matemáticas Luis Manuel Gutierrez Rivera}}
		{\emph{2012 -- Presente}}
	\begin{itemize*}
		\item Actualmente estoy cursando la carrera de Licenciatura en
		Ciencias Físico-Matemáticas.
	\end{itemize*}

\end{itemize}

\end{document}

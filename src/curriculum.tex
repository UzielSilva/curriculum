% resume.tex
% vim:set ft=tex spell:

\documentclass[10pt,letterpaper]{article}
\usepackage[letterpaper,margin=0.25in]{geometry}
\usepackage[utf8]{inputenc}
\usepackage{mdwlist}
\usepackage[T1]{fontenc}
\usepackage{textcomp}
\usepackage{tgpagella}
\pagestyle{empty}
\setlength{\tabcolsep}{0em}

\title{Currículum Vitae}

% indentsection style, used for sections that aren't already in lists
% that need indentation to the level of all text in the document
\newenvironment{indentsection}[1]%
{\begin{list}{}%
	{\setlength{\leftmargin}{#1}}%
	\item[]%
}
{\end{list}}

% opposite of above; bump a section back toward the left margin
\newenvironment{unindentsection}[1]%
{\begin{list}{}%
	{\setlength{\leftmargin}{-0.5#1}}%
	\item[]%
}
{\end{list}}

% format two pieces of text, one left aligned and one right aligned
\newcommand{\headerrow}[2]
{\begin{tabular*}{\linewidth}{l@{\extracolsep{\fill}}r}
	#1 &
	#2 \\
\end{tabular*}}

% make "C++" look pretty when used in text by touching up the plus signs
\newcommand{\CPP}
{C\nolinebreak[4]\hspace{-.05em}\raisebox{.22ex}{\footnotesize\bf ++}}

% and the actual content starts here
\begin{document}

\begin{center}
{\LARGE \textbf{Uziel Silva Espino}}

J. José de Lejarza \# 375\ \ \textbullet
\ \ Centro\ \ \textbullet
\ \ Morelia, Michoacán. México 58000
\\
443-195-1008\ \ \textbullet
\ \ uziel.silva.espino@gmail.com
\end{center}

\hrule
\vspace{-0.9em}
\subsection*{Cuentas personales}

\begin{indentsection}{\parindent}
\hyphenpenalty=1000
\begin{description*}
	\item[Skype:]
	uziel.silva.espino
	\item[GitHub:]
	UzielSilva
	
\end{description*}
\end{indentsection}

\hrule
\vspace{-0.4em}
\subsection*{Habilidades generales}

\begin{indentsection}{\parindent}
\hyphenpenalty=1000
\begin{description*}
	\item[Idiomas:] 
    Inglés (50\% - 70\%).
    
    \item[Metodologías Ágiles:]
    Scrum.
	
\end{description*}
\end{indentsection}

\hrule
\vspace{-0.4em}
\subsection*{Habilidades de programación}

\begin{indentsection}{\parindent}
\hyphenpenalty=1000
\begin{description*}
	\item[Habilidades generales:]
	Test Driven Development.
	
	\item[Herramientas de pruebas unitarias:]
	NUnit,
	JUnit,
	Jasmine,
    Karma,
    Phantom.js.
	
	\item[Lenguajes de programación:]
	VB6,
	VB.Net,
	Java(+JSP),
	JavaScript,
	C\#,
	C,
	F\#,
	HTML,
	PHP,
    Bash,
	Ruby. 
    
	\item[Servicios Web:]
	Java(RestEasy),
	PHP(Slim Framework).
	
	\item[Integración continua:]
	CruiseControl.NET,
    Travis CI. 
	
	\item[Bases de Datos:]
	MySQL 5.7(Con uso de funciones JSON),
    SQLite,
    MongoDB.
	
	\item[Javascript:]
	Angular.js,
    D3.js,
    Restangular,
    JQuery.
    
    \item[Herramientas de desarrollo:]
	Node.js,
    Grunt,
    Gulp,
    Bower,
    Yeoman,
    MSBuild.
    
    \item[Frameworks de desarrollo híbrido móvil:]
    PhoneGap/Cordova,
    Ionic Framework.
	
\end{description*}
\end{indentsection}

\hrule
\vspace{-0.4em}
\subsection*{Habilidades de automatización/testing}

\begin{indentsection}{\parindent}
\hyphenpenalty=1000
\begin{description*}

    \item[General skills:]
    Test Case Execution,
    Functional Testing,
    Exploratory Testing,
    Bug Fix Verification,
    Test Case Automation.
	
	\item[Testing tools:]
	TestStack.White,
	Selenium,
    Charles Proxy,
    Fiddler,
    Appium.
	
	\item[Code analysis:]
	StyleCop,
	Simian. 
\end{description*}
\end{indentsection}

\hrule
\vspace{-0.4em}
\subsection*{Experiencia.}

\begin{itemize}
	\parskip=0.1em

	\item
	\headerrow
		{\textbf{Scio}}
		{\textbf{http://sciodev.com/}}
	\\
	\headerrow
		{\emph{Application Developer 1C}}
		{\emph{2013 -- 2015}}
	\begin{itemize*}
		\item Test Driven Development. Poderosa práctica para programar profesionalmente
		que nos obliga a desarrollar código	funcional y mantenible, y reducir la probabilidad 
		de encontrar errores. Además nos ayuda con la documentación del código.
		\item Metodologías Ágiles, como Scrum. Scrum ayuda a la planificación realista de 
		proyectos. 
		\item Uso de repositorios como Git y SVN. Indispensable para mantener una misma versión de un proyecto.
	    \item Uso de herramientas de Integración Continua para automatizar las tareas comunes
	    para evaluar el código.
	\end{itemize*}
    
    \item
	\headerrow
		{\textbf{Freelance}}
		{\emph{2015 -- Presente}}
	\begin{itemize*}
		\item Desarrollo HTML5. A través de herramientas de Node.js, desarrollé aplicaciones HTML5 que cuentan con coverage de pruebas automatizadas y un servicio de Karma encargado de correrlas, así como RESTful API's en PHP y proceso de compresión de archivos js y css con manejadores de dependencias.
        \item Contexto de desarrollo móvil. He contribuído al desarrollo de aplicaciones móviles de consulta y compras en línea.
        \item Angular + D3.js. En un proyecto de ubicación geográfica automática desarrollé un sistema a través de algoritmos de geometría computacional. Explico mi solución aquí: \\ http://fismateros.blogspot.mx/2016/09/ensenando-una-computadora-usar-paint.html
	\end{itemize*}
    
    \item
	\headerrow
		{\textbf{uTest}}
		{\textbf{https://www.utest.com/}}
	\\
	\headerrow
		{\emph{Tester Funcional Freelance(Tester Plata)}}
		{\emph{Enero 2016 -- Presente}}
	\begin{itemize*}
		\item Metodología de ciclos de pruebas. En esta página he trabajado bajo el ambiente de ciclos de pruebas, haciendo reportes de errores eficientes y ejecución de casos de prueba, todo bajo fechas límite dadas por el cliente.
   	\end{itemize*}

\end{itemize}


\hrule
\vspace{-0.4em}
\subsection*{Educación Relevante}

\begin{itemize}
	\parskip=0.1em

	\item 
	\headerrow
		{\textbf{UMSNH (Universidad Michoacana de San Nicolas de Hidalgo).}}
		{\textbf{Morelia, Michoacán}}
	\\
	\headerrow
		{\emph{Facultad de Ciencias Físico-Matemáticas Luis Manuel Gutierrez Rivera}}
		{\emph{2012 -- Presente}}

\end{itemize}

\end{document}
